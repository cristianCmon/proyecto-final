% [Tamaño principal de la fuente del documento, tamaño del papel, con título (crea salto de página)]{Tipo de documento}
\documentclass[11pt, a4paper, titlepage]{article}
% Usa la codificación de fuentes T1, permite copiar correctamente textos con tildes
\usepackage[T1]{fontenc}
\usepackage[utf8]{inputenc}
% Usa fuentes Latin Modern (mejor calidad que la predeterminada)
\usepackage{lmodern}
% Traduce expresiones al español
\usepackage[spanish]{babel}
\usepackage{setspace}

% Añadido para Estilo Arial
\usepackage{helvet}
\renewcommand{\familydefault}{\sfdefault}

\usepackage{subcaption}
\usepackage{graphicx}
\usepackage{geometry}
\geometry{margin=2.5cm}
\graphicspath{ {./capturas/} }
% Permite utilizar labeling para listar de forma personalizada
\usepackage{scrextend}
% Permite centrar verticalmente m{}
\usepackage{array}
% Permite colorear texto
\usepackage{xcolor}
\usepackage[dvipsnames]{xcolor}
% Mejora índices
\usepackage{tocloft}


% Cambiar el título "Índice" a negrita y más grande
\renewcommand{\cfttoctitlefont}{\hspace*{\fill}\Huge\bfseries}
\renewcommand{\cftaftertoctitle}{\hspace*{\fill}} % Esto centra el título "Índice"

% Esto añade 3cm de espacio vertical antes de la primera sección
\setlength{\cftaftertoctitleskip}{3cm}

% 2. Añadir puntos guía (dots) hasta el número de página (si no aparecen)
\renewcommand{\cftsecleader}{\cftdotfill{\cftdotsep}}

% 3. Cambiar el formato de las secciones en el índice
\renewcommand{\cftsecfont}{\bfseries} % Secciones en negrita
\renewcommand{\cftsecpagefont}{\bfseries} % Números de página de secciones en negrita

% 4. Espaciado entre líneas en el índice
\setlength{\cftbeforesecskip}{1cm}



\title{\LARGE \textbf{Validaciones Zod/Zon} \\[2ex] \Large Afondamento nas Competencias Profesionais}
\author{\\[20ex]Cristian Fernández}
\date{Sábado, 21 de febrero de 2026}
% \date{\today}

\begin{document}
\maketitle

% Cambiado a interlineado 1.5
\onehalfspacing
\pagenumbering{gobble} % Oculta el número de página en el índice
\tableofcontents % Índice
\newpage
\pagenumbering{arabic} % Empieza a contar (1, 2, 3...) desde aquí

\section{Introducción}
\noindent \\El objetivo de esta práctica es aplicar la librería \textbf{Zon} en un proyecto en construcción que tiene 
como objetivo gestionar las actividades y reservas de un gimnasio. Dicho proyecto está configurado de la siguiente manera:\\
\begin{itemize}
\item \textbf{Backend API.} El backend está programado en \underline{python} usando el framework \underline{flask} con base de datos en \underline{MongoDB}.
Fue mi elección por 2 razones, la primera porque nunca había ``programado'' en python y la segunda porque flask tiene fama de ser muy agradable
de programar.
\item \textbf{Frontend Escritorio.} El frontend de escritorio está programado en \underline{Vue} y \underline{Electron} y la razón es porque las
especificaciones lo obligaban.
\item \textbf{Frontend Móvil.} El frontend móvil está programado en \underline{React Native}. Las opciones eran ésta o \underline{Flutter} y me decidí
por React por estar ya familiarizado por una práctica anterior y porque es muy sencillo compilarlo en máquina real.\\
\end{itemize} 
\section{¿Por qué implementarlo?}
\noindent \\La razón principal para implementar esta librería es sencilla: \textbf{SIMPLIFICACIÓN}.\\\\
\textbf{Zon} nos permitirá definir esquemas declarativos y legibles y validar datos en runtime. Esto nos proporcionará ventajas tales como
un código menos repetitivo y más simple, tener las reglas centralizadas en un solo esquema, mensajes de error más consistentes, etc.\\\\
\textit{\underline{La librería se implementará en el Backend}}, y específicamente es la siguiente:
\begin{itemize}
    \item \textit{zon (pip install zon)}
\end{itemize}
\noindent \\Se puede encontrar en https://pypi.org/
\newpage

\section{¿Qué es Zod/Zon?}
\noindent \\\textbf{Zod/Zon} es una biblioteca de validación de esquemas, diseñada para garantizar la integridad y el tipo de los datos en tiempo de ejecución
de forma sencilla y expresiva. Permite definir estructuras de datos (strings, números, objetos, arrays) una única vez.\\\\
Características principales:\\
\begin{itemize}
    \item \textbf{Seguridad de tipos.} Ofrece una inferencia de tipos automática y precisa, asegurando que los datos cumplan con las reglas definidas.
    \item \textbf{Validación en tiempo de ejecución.} Verifica que los datos recibidos (por ejemplo, desde una API o un formulario) cumplan con el esquema.
    \item \textbf{Sintaxis declarativa.} Se definen los datos mediante esquemas, por ejemplo, z.string().email(), lo que facilita la lectura y el mantenimiento.
    \item \textbf{Mensajes de error personalizables.} Permite generar errores claros y personalizados si los datos no coinciden con el esquema.
\end{itemize}
\newpage

\section{Implementación}
\noindent \\Crearemos un nuevo archivo en nuestro backend en el que alojar los esquemas de validación para el registro de usuarios y el login.\\\\
Así quedaría el código.\\
\begin{figure}[hbt!]
    \centering
    \includegraphics[width=0.9\textwidth]{Screenshot_1.png}
    \caption{Archivo \textit{validaciones.py}, los esquemas son muy legibles}
\end{figure}
\noindent \\\\\\Debemos importar los esquemas en nuestra API.\\
\begin{figure}[hbt!]
    \centering
    \includegraphics[width=0.9\textwidth]{Screenshot_2.png}
\end{figure}
\newpage
\noindent \\Echemos un vistazo al comienzo del endpoint de registro.\\
\begin{figure}[hbt!]
    \centering
    \includegraphics[width=1\textwidth]{Screenshot_3.png}
\end{figure}
\noindent \\\\Ahora observemos el comienzo del endpoint de login.\\
\begin{figure}[hbt!]
    \centering
    \includegraphics[width=1\textwidth]{Screenshot_4.png}
\end{figure}
\noindent \\\\En el comienzo de ambos endpoints se puede observar algo muy interesante y es la tremenda simplificación de código, ya que todo la parte de validación
se reduce a 4 líneas.
\newpage
\noindent \\Veamos en la siguiente página el proceso de registrar un nuevo usuario dentro del frontend de escritorio.\\
\begin{figure}[hbt!]
    \centering
    \includegraphics[width=0.5\textwidth]{Screenshot_5.png}
    \caption{Vista en desarrollo}
\end{figure}
\noindent \\\\Comprobamos el nuevo registro en la base de datos\\
\begin{figure}[hbt!]
    \centering
    \includegraphics[width=0.7\textwidth]{Screenshot_6.png}
\end{figure}
\noindent \\\\Como se puede apreciar, la validación del formulario ha sido satisfactoria.
\newpage

\end{document}