% [Tamaño principal de la fuente del documento, tamaño del papel, con título (crea salto de página)]{Tipo de documento}
\documentclass[11pt, a4paper, titlepage]{article}
% Usa la codificación de fuentes T1, permite copiar correctamente textos con tildes
\usepackage[T1]{fontenc}
\usepackage[utf8]{inputenc}
% Usa fuentes Latin Modern (mejor calidad que la predeterminada)
\usepackage{lmodern}
% Traduce expresiones al español
\usepackage[spanish]{babel}
\usepackage{setspace}

% Añadido para Estilo Arial
\usepackage{helvet}
\renewcommand{\familydefault}{\sfdefault}

\usepackage{subcaption}
\usepackage{graphicx}
\usepackage{geometry}
\geometry{margin=2.5cm}
\graphicspath{ {./capturas/} }
% Permite utilizar labeling para listar de forma personalizada
\usepackage{scrextend}
% Permite centrar verticalmente m{}
\usepackage{array}
% Permite colorear texto
\usepackage{xcolor}
\usepackage[dvipsnames]{xcolor}
% Mejora índices
\usepackage{tocloft}


% Cambiar el título "Índice" a negrita y más grande
\renewcommand{\cfttoctitlefont}{\hspace*{\fill}\Huge\bfseries}
\renewcommand{\cftaftertoctitle}{\hspace*{\fill}} % Esto centra el título "Índice"

% Esto añade 3cm de espacio vertical antes de la primera sección
\setlength{\cftaftertoctitleskip}{3cm}

% 2. Añadir puntos guía (dots) hasta el número de página (si no aparecen)
\renewcommand{\cftsecleader}{\cftdotfill{\cftdotsep}}

% 3. Cambiar el formato de las secciones en el índice
\renewcommand{\cftsecfont}{\bfseries} % Secciones en negrita
\renewcommand{\cftsecpagefont}{\bfseries} % Números de página de secciones en negrita

% 4. Espaciado entre líneas en el índice
\setlength{\cftbeforesecskip}{1cm}



\title{\LARGE \textbf{Bcrypt y JWT} \\[2ex] \Large Programación de Servicios y Procesos}
\author{\\[20ex]Cristian Fernández}
\date{Viernes, 20 de febrero de 2026}
% \date{\today}

\begin{document}
\maketitle

% Cambiado a interlineado 1.5
\onehalfspacing
\pagenumbering{gobble} % Oculta el número de página en el índice
\tableofcontents % Índice
\newpage
\pagenumbering{arabic} % Empieza a contar (1, 2, 3...) desde aquí

\section{Introducción}
\noindent \\El objetivo de esta práctica es aplicar las librerías \textbf{Bcrypt} y \textbf{JWT} en una proyecto en construcción que tiene 
como objetivo gestionar las actividades y reservas de un gimnasio. Dicho proyecto está configurado de la siguiente manera:\\
\begin{itemize}
\item \textbf{Backend API.} El backend está programado en \underline{python} usando el framework \underline{flask} con base de datos en \underline{MongoDB}.
Fue mi elección por 2 razones, la primera porque nunca había ``programado'' en python y la segunda porque flask tiene fama de ser muy agradable
de programar.
\item \textbf{Frontend Escritorio.} El frontend de escritorio está programado en \underline{Vue} y \underline{Electron} y la razón es porque las
especificaciones lo obligaban.
\item \textbf{Frontend Móvil.} El frontend móvil está programado en \underline{React Native}. Las opciones eran ésta o \underline{Flutter} y me decidí
por React por estar ya familiarizado por una práctica anterior y porque es muy sencillo compilarlo en máquina real.\\
\end{itemize} 
\section{¿Por qué implementarlos?}
\noindent \\La razón principal para implementar estas librerías en el proyecto es simple: \textbf{SEGURIDAD}.\\\\
\textbf{Bcrypt} nos permitirá hashear las contraseñas de los usuarios en la base de datos, dando una capa extra de seguridad en comparación con la 
encriptación. Si un ataque mal intencionado a la base de datos expusiese la información del usuario, el atacante se encontraría con una contraseña
hasheada imposible de descrifrar.\\\\
\textbf{JWT} nos posibilitará crear sesiones para los diferentes usuarios de la aplicación en los diferentes frontends mediante la generación de tokens.
Dichos tokens, de duración variable, nos permitirán autenticar, autorizar e intercambiar información en los diferentes endpoints de la aplicación.\\\\
\textit{\underline{Las librerías se implementarán en el Backend}}, y específicamente son las siguientes:
\begin{itemize}
    \item \textit{flask-bcrypt}
    \item \textit{flask-jwt-extended}
\end{itemize}
\noindent \\Ambas se pueden encontrar en https://pypi.org/
\newpage

\section{¿Qué son Bcrypt y JWT?}
\noindent \\\textbf{Bcrypt} es una función de hashing (derivación de claves) diseñada para proteger contraseñas, basada en el cifrado de bloque Blowfish. A diferencia
de algoritmos rápidos como SHA256, bcrypt es intencionalmente lento, lo que lo hace altamente resistente a ataques de fuerza bruta y precomputados.\\\\
Características principales:\\
\begin{itemize}
    \item \textbf{Hash, no encriptación.} Bcrypt transforma una contraseña en una cadena de caracteres ininteligible (un hash) de una sola vía. No se puede "desencriptar"
    para obtener la contraseña original, solo verificarla. 
    \item \textbf{Salting (Sal).} Bcrypt genera automáticamente una "sal" (una cadena aleatoria) para cada contraseña antes de hashearla. Esto asegura que, si dos usuarios
    usan la misma contraseña ("123456"), sus hashes almacenados sean completamente diferentes, protegiendo contra ataques de fuerza bruta.
\end{itemize}
\noindent \\\\\\\textbf{JWT (JSON Web Token)} es un estándar abierto que define una forma compacta y autónoma para transmitir información segura entre partes
como un objeto JSON. Es ampliamente utilizado para la autenticación y autorización en aplicaciones web, permitiendo validar la identidad del usuario sin necesidad de
almacenar sesiones en el servidor.\\\\
Características principales:\\
\begin{itemize}
    \item \textbf{Autónomo (Stateless).} El token contiene toda la información necesaria sobre el usuario, eliminando la necesidad de consultar la
    base de datos en cada petición.
    \item \textbf{Firmado digitalmente.} Los JWT se firman (usando secretos o pares de claves pública/privada) para garantizar la integridad y autenticidad de los datos.
\end{itemize}
\newpage

\section{Implementación}
\noindent \\Lo primero que debemos hacer es importar las librerías correspondientes en nuestra API.
\begin{figure}[hbt!]
    \centering
    \includegraphics[width=1\textwidth]{Screenshot_1.png}
    \caption{CORS será necesaria para que haya comunicación con los frontend}
\end{figure}
\noindent \\\\\\Empecemos por mostrar el endpoint de registro.
\begin{figure}[hbt!]
    \centering
    \includegraphics[width=1\textwidth]{Screenshot_2.png}
    \caption{Parte 1 - Parece que ha habido cambios en el sistema de validación...}
\end{figure}
\newpage
\begin{figure}[hbt!]
    \centering
    \includegraphics[width=1\textwidth]{Screenshot_3.png}
    \caption{Parte 2}
\end{figure}
\noindent \\\\En la línea 94 podemos observar cómo asignamos a una variable la contraseña obtenida en la cabecera de la petición a la hora
de hacer un registro. Posteriormente, en la línea 95, aplicamos el hasheo a la constraseña y la almacenamos en una nueva variable que, posteriormente
en la línea 99, la incluiremos como parte del objeto \textit{`nuevoUsuario'} que almacenaremos en la base de datos en la línea 110.
\newpage
\noindent El trabajo en el endpoint de registro ya está hecho, pasemos ahora al endpoint de login.
\begin{figure}[hbt!]
    \centering
    \includegraphics[width=1\textwidth]{Screenshot_4.png}
\end{figure}
\noindent \\\\En la línea 138 podemos comprobar que, si existe en la base de datos el nombre de usuario introducido y, si la contraseña hasheada
almacenada de ese usuario se compara con la \textit{``contraseña plana''} obtenida del login y devuelve verdadero el login será satisfactorio y 
crearemos un token. De lo contrario devolverá un error 401.\\\\
Los métodos clave son \textit{check\_password\_hash} y \textit{create\_access\_token} de las librerías bcrypt y jwt respectivamente.
\newpage
\noindent Vamos a logearnos en la aplicación con un usuario ya creado en MongoDB.
\begin{figure}[hbt!]
    \centering
    \includegraphics[width=0.8\textwidth]{Screenshot_5.png}
    \caption{Como se puede apreciar, la contraseña está perfectamente hasheada}
\end{figure}
\noindent \\\\Veamos el login del frontend de escritorio.
\begin{figure}[hbt!]
    \centering
    \includegraphics[width=1\textwidth]{Screenshot_6.png}
    \caption{Vista en desarrollo}
\end{figure}
\newpage
\noindent Si el logeo es satisfactorio, entraremos en la aplicación.
\begin{figure}[hbt!]
    \centering
    \includegraphics[width=1\textwidth]{Screenshot_7.png}
    \caption{Vista en desarrollo}
\end{figure}
\noindent \\\\Como se puede apreciar, en el Session Storage queda almacenado el token del usuario, su nombre y su rol.
\newpage

\end{document}