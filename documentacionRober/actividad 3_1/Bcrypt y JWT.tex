% [Tamaño principal de la fuente del documento, tamaño del papel, con título (crea salto de página)]{Tipo de documento}
\documentclass[11pt, a4paper, titlepage]{article}
% Usa la codificación de fuentes T1, permite copiar correctamente textos con tildes
\usepackage[T1]{fontenc}
\usepackage[utf8]{inputenc}
% Usa fuentes Latin Modern (mejor calidad que la predeterminada)
\usepackage{lmodern}
% Traduce expresiones al español
\usepackage[spanish]{babel}
\usepackage{setspace}

% Añadido para Estilo Arial
\usepackage{helvet}
\renewcommand{\familydefault}{\sfdefault}

\usepackage{subcaption}
\usepackage{graphicx}
\usepackage{geometry}
\geometry{margin=2.5cm}
\graphicspath{ {./capturas/} }
% Permite utilizar labeling para listar de forma personalizada
\usepackage{scrextend}
% Permite centrar verticalmente m{}
\usepackage{array}
% Permite colorear texto
\usepackage{xcolor}
\usepackage[dvipsnames]{xcolor}
% Mejora índices
\usepackage{tocloft}


% Cambiar el título "Índice" a negrita y más grande
\renewcommand{\cfttoctitlefont}{\hspace*{\fill}\Huge\bfseries}
\renewcommand{\cftaftertoctitle}{\hspace*{\fill}} % Esto centra el título "Índice"

% Esto añade 3cm de espacio vertical antes de la primera sección
\setlength{\cftaftertoctitleskip}{3cm}

% 2. Añadir puntos guía (dots) hasta el número de página (si no aparecen)
\renewcommand{\cftsecleader}{\cftdotfill{\cftdotsep}}

% 3. Cambiar el formato de las secciones en el índice
\renewcommand{\cftsecfont}{\bfseries} % Secciones en negrita
\renewcommand{\cftsecpagefont}{\bfseries} % Números de página de secciones en negrita

% 4. Espaciado entre líneas en el índice
\setlength{\cftbeforesecskip}{1cm}



\title{\LARGE \textbf{Bcrypt y JWT} \\[2ex] \Large Programación de Servicios y Procesos}
\author{\\[20ex]Cristian Fernández}
\date{Viernes, 20 de febrero de 2026}
% \date{\today}

\begin{document}
\maketitle

% Cambiado a interlineado 1.5
\onehalfspacing
\pagenumbering{gobble} % Oculta el número de página en el índice
\tableofcontents % Índice
\newpage
\pagenumbering{arabic} % Empieza a contar (1, 2, 3...) desde aquí

\section{Introducción}
\noindent El objetivo de esta práctica es aplicar las librerías \textbf{Bcrypt} y \textbf{JWT} en una proyecto en construcción que tiene 
como objetivo gestionar las actividades y reservas de un gimnasio. Dicho proyecto está configurado de la siguiente manera:\\
\begin{itemize}
\item \textbf{Backend API.} El backend está programado en \underline{python} usando el framework \underline{flask} y fue mi elección por 2 razones,
la primera porque nunca había ``programado'' en python y la segunda porque flask tiene fama de ser muy agradable de programar.
\item \textbf{Frontend Escritorio.} El frontend de escritorio está programado en \underline{Vue} y \underline{Electron} y la razón es porque las
especificaciones lo obligaban.
\item \textbf{Frontend Móvil.} El frontend móvil está programado en \underline{React Native}. Las opciones eran ésta o \underline{Flutter} y me decidí
por React por estar ya familiarizado por una práctica anterior y porque es muy sencillo compilarlo en máquina real.\\
\end{itemize} 
\section{¿Por qué implementarlos?}
\noindent La razón principal para implementar estas librerías en el proyecto es simple: \textbf{SEGURIDAD}.\\\\
\textbf{Bcrypt} nos permitirá hashear las contraseñas de los usuarios en la base de datos, dando una capa extra de seguridad en comparación con la 
encriptación. Si un ataque mal intencionado a la base de datos expusiese la información del usuario, el atacante se encontraría con una contraseña
hasheada imposible de descrifrar.\\\\
\textbf{JWT} nos posibilitará crear sesiones para los diferentes usuarios de la aplicación en los diferentes frontends mediante la generación de tokens.
Dichos tokens, de duración variable, nos permitirá autenticar, autorizar e intercambiar información en los diferentes endpoints de la aplicación.\\\\
\textit{\underline{Las librerías se implementarán en el Backend}}, y específicamente son las siguientes:
\begin{itemize}
    \item \textit{flask-bcrypt}
    \item \textit{flask-jwt-extended}
\end{itemize}
\noindent \\Ambas se pueden encontrar en https://pypi.org/
\newpage

\section{¿Qué son Bcrypt y JWT?}
\newpage


\section{Implementación}
\newpage

\end{document}